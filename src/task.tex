\section{Порядок выполнения работы}
\begin{enumerate}
    \item Выбрать вариант для выполнения работы по формулам:
    $V_1 = 1 + (N \mathrm{mod} 5)$, $V_2 = 6 + (N \mathrm{mod} 5)$,
    где $V_1$ и $V_2$ --- номера вариантов;
    $N$ --- сумма количества букв в фамилии и имени студента;
    $\mathrm{mod}$ --- операция взятия остатка от деления.

    \item На всех адаптерах всех компьютеров в топологии,
    представленной в варианте, настроить IPv4-адреса (и IPv6, если необходимо).
    IPv4-адрес выбирается следующим образом: \texttt{A.B.X.Y/M},
    где \texttt{A} --- количество букв в имени студента;
    \texttt{B} --- количество букв в фамилии студента;
    \texttt{X}, \texttt{Y} --- числа, выбираемые студентом самостоятельно;
    \texttt{M} --- маска подсети (выбирается максимально длинная маска для обеспечения связности в сети).
    IPv6-адрес формируется из IPv4-адреса в соответствии с нотацией перевода адресов из IPv4 в IPv6.
    Например:

    IPv4: \texttt{10.10.12.11}

    IPv6: \texttt{0:0:0:0:0:ffff:a0a:c0b} (или иначе: \texttt{::ffff:10.10.12.11}).

    \item На всех компьютерах настроить таблицы маршрутизации таким образом,
    чтобы обеспечивалась полная сетевая доступность
    (каждый компьютер должен ,,пинговаться'' с каждого другого компьютера).

    \item Изучить Linux-утилиту \texttt{nc} (или ее аналоги: \texttt{netcat}, \texttt{ncat}, \texttt{pnetcat}).
    Запустить ее в режиме клиента на машине А и в режиме сервера --- на машине Б,
    используя для передачи произвольный порт (машины А и Б должны быть максимально удалены друг от друга).
    Передать в виде текстового сообщения свое имя от Б к А.

    \item Изучить назначение Linux-утилиты \texttt{iptables}
    (например, тут: \texttt{www.k-max.name/linux/iptables-v-primerax})
    и создать на компьютерах А и/или Б простейший Firewall (межсетевой экран)
    с помощью этой утилиты следующим образом:

    \begin{itemize}
        \item Запретить передачу только тех пакетов, которые отправлены на TCP-порт,
        заданный в настройках утилиты \texttt{nc}.

        \item Запретить прием только тех пакетов, которые отправлены с UDP-порта утилиты \texttt{nc}.

        \item Запретить передачу только тех пакетов, которые отправлены с IP-адреса компьютера А.

        \item Запретить прием только тех пакетов, которые отправлены на IP-адрес компьютера Б.

        \item Запретить прием и передачу ICMP-пакетов, размер которых превышает 1000 байт,
        а поле TTL при этом меньше 10.
    \end{itemize}

    \item Убедиться с помощью команды \texttt{ping}, \texttt{traceroute} или \texttt{nc},
    что настроенные правила фильтрации \texttt{iptables} работоспособны,
    для чего нужно сначала попытаться передать запрещенный пакет, а затем разрешенный.

    \item Выполнение пунктов с 1 по 6 позволяет получить оценку ,,3E''.
    Для получения более высокой оценки необходимо выполнить дополнительные задания в соответствии с вариантом
    $V_1$ и $V_2$.
    Дополнительные задания могут потребовать изменения настроек, сделанных на шагах со 2 по 6-й.
\end{enumerate}
